\documentclass[]{book}
\usepackage{lmodern}
\usepackage{amssymb,amsmath}
\usepackage{ifxetex,ifluatex}
\usepackage{fixltx2e} % provides \textsubscript
\ifnum 0\ifxetex 1\fi\ifluatex 1\fi=0 % if pdftex
  \usepackage[T1]{fontenc}
  \usepackage[utf8]{inputenc}
\else % if luatex or xelatex
  \ifxetex
    \usepackage{mathspec}
  \else
    \usepackage{fontspec}
  \fi
  \defaultfontfeatures{Ligatures=TeX,Scale=MatchLowercase}
\fi
% use upquote if available, for straight quotes in verbatim environments
\IfFileExists{upquote.sty}{\usepackage{upquote}}{}
% use microtype if available
\IfFileExists{microtype.sty}{%
\usepackage{microtype}
\UseMicrotypeSet[protrusion]{basicmath} % disable protrusion for tt fonts
}{}
\usepackage[margin=1in]{geometry}
\usepackage{hyperref}
\hypersetup{unicode=true,
            pdftitle={A collection of notes on R, Git and statistics},
            pdfauthor={Olalla Diaz-Yañez},
            pdfborder={0 0 0},
            breaklinks=true}
\urlstyle{same}  % don't use monospace font for urls
\usepackage{natbib}
\bibliographystyle{apalike}
\usepackage{longtable,booktabs}
\usepackage{graphicx,grffile}
\makeatletter
\def\maxwidth{\ifdim\Gin@nat@width>\linewidth\linewidth\else\Gin@nat@width\fi}
\def\maxheight{\ifdim\Gin@nat@height>\textheight\textheight\else\Gin@nat@height\fi}
\makeatother
% Scale images if necessary, so that they will not overflow the page
% margins by default, and it is still possible to overwrite the defaults
% using explicit options in \includegraphics[width, height, ...]{}
\setkeys{Gin}{width=\maxwidth,height=\maxheight,keepaspectratio}
\IfFileExists{parskip.sty}{%
\usepackage{parskip}
}{% else
\setlength{\parindent}{0pt}
\setlength{\parskip}{6pt plus 2pt minus 1pt}
}
\setlength{\emergencystretch}{3em}  % prevent overfull lines
\providecommand{\tightlist}{%
  \setlength{\itemsep}{0pt}\setlength{\parskip}{0pt}}
\setcounter{secnumdepth}{5}
% Redefines (sub)paragraphs to behave more like sections
\ifx\paragraph\undefined\else
\let\oldparagraph\paragraph
\renewcommand{\paragraph}[1]{\oldparagraph{#1}\mbox{}}
\fi
\ifx\subparagraph\undefined\else
\let\oldsubparagraph\subparagraph
\renewcommand{\subparagraph}[1]{\oldsubparagraph{#1}\mbox{}}
\fi

%%% Use protect on footnotes to avoid problems with footnotes in titles
\let\rmarkdownfootnote\footnote%
\def\footnote{\protect\rmarkdownfootnote}

%%% Change title format to be more compact
\usepackage{titling}

% Create subtitle command for use in maketitle
\newcommand{\subtitle}[1]{
  \posttitle{
    \begin{center}\large#1\end{center}
    }
}

\setlength{\droptitle}{-2em}
  \title{A collection of notes on R, Git and statistics}
  \pretitle{\vspace{\droptitle}\centering\huge}
  \posttitle{\par}
  \author{Olalla Diaz-Yañez}
  \preauthor{\centering\large\emph}
  \postauthor{\par}
  \predate{\centering\large\emph}
  \postdate{\par}
  \date{2018-03-03}

\usepackage{booktabs}
\usepackage{amsthm}
\makeatletter
\def\thm@space@setup{%
  \thm@preskip=8pt plus 2pt minus 4pt
  \thm@postskip=\thm@preskip
}
\makeatother

\usepackage{amsthm}
\newtheorem{theorem}{Theorem}[chapter]
\newtheorem{lemma}{Lemma}[chapter]
\theoremstyle{definition}
\newtheorem{definition}{Definition}[chapter]
\newtheorem{corollary}{Corollary}[chapter]
\newtheorem{proposition}{Proposition}[chapter]
\theoremstyle{definition}
\newtheorem{example}{Example}[chapter]
\theoremstyle{definition}
\newtheorem{exercise}{Exercise}[chapter]
\theoremstyle{remark}
\newtheorem*{remark}{Remark}
\newtheorem*{solution}{Solution}
\begin{document}
\maketitle

{
\setcounter{tocdepth}{1}
\tableofcontents
}
\hypertarget{a-work-in-progress}{%
\chapter{A work in progress}\label{a-work-in-progress}}

This is a collection of notes (a book?) with useful resources and
instructions related to R, RStudio, Git and Statistics with R. Right now
it mainly works as a personal repository. But I hope that litle by litle
it will have more meaninful content with detailed instructions to more
complex concepts and examples.

{A collection of notes on R, Git and statistics} by Olalla Díaz-Yáñez is
licensed under a Creative Commons Attribution-NonCommercial 4.0
International License.

\hypertarget{intro}{%
\chapter{Introduction}\label{intro}}

\hypertarget{what-is-r}{%
\section{What is R}\label{what-is-r}}

R is a system for statistical computation and graphics. It consists of a
language plus a run-time environment with graphics, a debugger, access
to certain system functions, and the ability to run programs stored in
script files. To be able to use and understand R you will also need to
know some basic concepts that all programming depends on. But do not be
scared!

\href{https://cran.r-project.org/doc/FAQ/R-FAQ.html\#What-is-R_003f}{Here}
you can read a longer deffinition of what is R.

\textbf{R vs.~Python}

I really like
\href{http://ucanalytics.com/blogs/r-vs-python-comparison-and-awsome-books-free-pdfs-to-learn-them/}{this}
comparison of python and R. The idea is that R is Batman and Python is
Superman. Batman (R) does better detective work, has a more developed
intelligence, or in other words Batmand is more brain than muscles. On
the other hand Superman (Python) has muscle power and super strength,
you could consider him more elegant, but in general words is more
muscles than brain.

FUN FACT: The ``Python'' programming language name derived from the
series
\href{https://en.wikipedia.org/wiki/Monty_Python\%27s_Flying_Circus}{Monty
Python's Flying Circus}.

\hypertarget{why-r}{%
\subsection{Why R}\label{why-r}}

There are several reason why R:

\begin{itemize}
\tightlist
\item
  it´s free
\item
  it´s well-documented and has an amazing user community
\item
  it runs almost everywhere
\item
  it has a large user base among researchers, data scientists, companies
\item
  it has a extensive library of packages helping to solve different
  tasks
\item
  it´s not a black box
\end{itemize}

The best way to learn a tool is to use it for something useful, for
example analyze data. Thats why this is the tool prefered in our
courses. The goal it is not to master the tool or actually teach you R
but to have the enough knowledge to be able to find your way to suceed
in your goals and to have a basic knowledge that will aloow you to
explore independtly and find the solutions that you need.

\hypertarget{what-is-rstudio}{%
\section{What is RStudio}\label{what-is-rstudio}}

\href{https://www.rstudio.com/products/RStudio/}{RStudio} is an
integrated development environment (IDE) for R. It includes a console,
syntax-highlighting editor that supports direct code execution, as well
as tools for plotting, history, debugging and workspace management.

\hypertarget{why-rstudio}{%
\subsection{Why RStudio}\label{why-rstudio}}

My friend usually sais that we have not come to this world to suffer,
using Rstudio instead of just R makes your life easier, so it will avoid
some unnecesary suffering. There is nothing wrong of just using R,
without RStudio, and actually some people prefer to learn R without
RStudio, not me. Some of the reasons why I prefer to use RStudio are:

\begin{itemize}
\tightlist
\item
  window docking (all necessary things in one window)
\item
  full-featured text editor
\item
  tab-completion of filenames, function names and arguments (you do not
  need to remember everything)
\item
  Rmarkdown and knitr integration
\end{itemize}

\hypertarget{what-is-git}{%
\section{What is Git}\label{what-is-git}}

\href{http://git-scm.com}{Git} is a \textbf{version control system}.
Probably the best ever description og Git is whaty XXX wrote: ``it is as
the ``Track Changes'' features from Microsoft Word on steroids''. It was
created to help groups of developers to deal with big and complex
projects. For a data science user, it is basically a clever way to ovoid
having hundred ``final version'' files as described
\href{http://phdcomics.com/comics.php?f=1531}{here}.

It is both benefitial when working alone as you can delete that
not-so-clever code that you wrote and never used, without been scared,
as if future you need that piced of code you will be able to go back and
take it. But Git benefits increase exponinetally when you include
collaborators in the equation, use git it is a smart way to collaborate
and to be up-to-date with each others work and at the same time have a
version control of your and others people work. Some people think that
hthe Git-pain is only worth it when collaborating, but even in that
case, I am sure you are going to work with others in some point, and
avoid taken that into consideration from the beginning will mean a delay
in the implementation of it in your workflow and a higher pain than just
Git-pain.

\hypertarget{what-is-github}{%
\section{What is GitHub}\label{what-is-github}}

A way of hosting your work online.

\hypertarget{what-it-is-in-for-me}{%
\section{What it is in for me?}\label{what-it-is-in-for-me}}

Maybe you are still wondering: what it is really in for me? I just
wanted to do my statistical anaylisis and be done with it, do really all
the gains possibly justify the inevitable pain of start using R,
Rstudio, GitHub and Git?\ldots{}

My personal experience is that doing things from the beginning with the
correct aproach, although painful, may avoid a bigger pain in the
future. As soon as you get into your best workflow easier for you would
be to do things right and realised early about mistakes. Of course your
workflow won't be a static thing, you will continiouly learn new
aproaches and techniques that will improve the way you do things. Saying
that I also think that one thing is what it is the best workflow for
each of us as individuals (if you like to write your essays on a paper
and then in a word document, thats fine for me), but a different thing
is what is the best way to collaborate and work with others. In the
first case you chose, in the second case chosing should be always on
what is the best for the team-work, meaning higher productivity, less
chances of errors, easier collaboration etc. And yes, you are going to
work with others most of the times. There is were Git and RStudio are
going to make your life easier!

\hypertarget{applications}{%
\chapter{Applications}\label{applications}}

Some \emph{significant} applications of the tools presented

\hypertarget{example-one}{%
\section{Example one}\label{example-one}}

\hypertarget{example-two}{%
\section{Example two}\label{example-two}}

\hypertarget{sessions}{%
\chapter{Sessions}\label{sessions}}

\hypertarget{session-0}{%
\section{Session 0}\label{session-0}}

\textbf{1. Install R and R studio}

\begin{itemize}
\item
  Install pre-compiled binary of R for your operating system:
  \url{https://cloud.r-project.org}
\item
  Install Preview version RStudio Desktop:
  \url{https://www.rstudio.com/products/rstudio/download/preview/}
\end{itemize}

If you have a pre-existing installation of R and/or RStudio, it is
highly recommend that you reinstall both. You will face more dificulties
if you run an old softwared version. If you upgrade R, you will need to
update any packages you have installed.

\textbf{2. Test it}

Launch RStudio, you should see something simmilar to
\href{https://www.flickr.com/photos/xmodulo/22093054381}{this} but a bit
emptier as you have not written anythign yet. Put your cursor in the
pane labelled Console, which is where you interact with the live R
process. In console, write something like: 2 + 1 and press return and
you should get a 3. If you get a 3, you've succeeded in the installation
of R and RStudio, congrats!

\textbf{Some extra resources:}

\url{https://cran.r-project.org/doc/manuals/R-admin.html}

\url{https://cran.r-project.org/doc/FAQ/R-FAQ.html\#What-is-R_003f}

\url{https://cran.r-project.org/doc/FAQ/R-FAQ.html}

\hypertarget{session-1}{%
\section{Session 1}\label{session-1}}

\hypertarget{learning-objectives}{%
\subsection{Learning objectives}\label{learning-objectives}}

\begin{itemize}
\tightlist
\item
  Learn what is programming
\item
  Learn what is R
\item
  Learn the basics of RStudio
\item
  Learn the basics of R:

  \begin{itemize}
  \tightlist
  \item
    Variable assigment
  \item
    Basic data types
  \item
    Vectors
  \item
    Data frames
  \end{itemize}
\end{itemize}

\hypertarget{contents}{%
\subsection{Contents}\label{contents}}

\textbf{Basics of R:}

\begin{itemize}
\tightlist
\item
  Program
\item
  Language: Not compiled, simple sintaxys
\item
  R ``flow'':

  \begin{itemize}
  \tightlist
  \item
    Variables, data, functions, results, etc, are stored in the active
    memory of the computer in the form of objects which have a name.
  \item
    You can do actions on these objects with operators (arithmetic,
    logical, comparison, . . .) and functions (which are themselves
    objects).
  \end{itemize}
\end{itemize}

\textbf{Practice}

\begin{itemize}
\tightlist
\item
  Using Rstudio

  \begin{itemize}
  \tightlist
  \item
    Open a project
  \item
    Organize your folder (rstudio projet / Code / Data / Figures / )
    \textbf{Super tip of the day!}
  \item
    Console
  \item
    Files
  \item
    Script
  \item
    \href{https://support.rstudio.com/hc/en-us/articles/200711853-Keyboard-Shortcuts}{Keyboard
    shortcuts}
  \end{itemize}
\item
  Variable assigment: the ``assign'' operator

  \begin{itemize}
  \tightlist
  \item
    concept overwiting a variable (s in the active memory, not the data
    on the disk)
  \item
    Note that R is case sensitive!
  \item
    ``\#'' is used for comments
  \end{itemize}
\item
  Basic data types

  \begin{itemize}
  \tightlist
  \item
    Decimals values like 4.5 are called numerics.
  \item
    Natural numbers like 4 are called integers. Integers are also
    numerics.
  \item
    Boolean values (TRUE or FALSE) are called logical.
  \item
    Text (or string) values are called characters. The quotation marks
    indicate that is a character.
  \end{itemize}
\item
  Vectors

  \begin{itemize}
  \tightlist
  \item
    create vectors
  \item
    name a vector
  \item
    select elements from the vector
  \item
    compare different vectors
  \item
    combine vectors
  \end{itemize}
\item
  Data frames

  \begin{itemize}
  \tightlist
  \item
    creating a data frame
  \item
    quick look to the data frame: structure, rownames, number of
    columns, number of rows, summary,
  \item
    select data fram elements
  \item
    subset in a data frame
  \item
    ordering
  \item
    sorting
  \end{itemize}
\end{itemize}

\textbf{The code of this session} can be found
\href{https://github.com/oldiya/DataScience/blob/master/scripts/session1.R}{here}

\hypertarget{carry-on-learning}{%
\subsection{Carry on learning}\label{carry-on-learning}}

\href{https://www.datacamp.com/courses/free-introduction-to-r?utm_source=adwords_ppc\&utm_campaignid=898687156\&utm_adgroupid=48303643819\&utm_device=c\&utm_keyword=\&utm_matchtype=b\&utm_network=g\&utm_adpostion=1t1\&utm_creative=229335520231\&utm_targetid=dsa-377762271983\&utm_loc_interest_ms=\&utm_loc_physical_ms=1005620\&gclid=EAIaIQobChMIkOeW3InG2QIVdijTCh23qgmmEAAYASAAEgIJL_D_BwE}{Data
camp (still) free basic r course}

\href{https://www.rstudio.com/wp-content/uploads/2016/10/r-cheat-sheet-3.pdf}{Base
R Cheat Sheet}

\hypertarget{session-2}{%
\section{Session 2}\label{session-2}}

\hypertarget{learning-objectives-1}{%
\subsection{Learning objectives}\label{learning-objectives-1}}

\begin{itemize}
\tightlist
\item
  Learn the basics of R:

  \begin{itemize}
  \tightlist
  \item
    Variable assigment (review)
  \item
    Basic data types (review)
  \item
    Vectors (review)
  \item
    Data frames (review \& new)
  \item
    Lists
  \item
    Matrices
  \item
    Factors
  \end{itemize}
\item
  Basic math functions
\item
  Basic visualizations of the data
\item
  Basic statistics
\item
  Using Libraries
\item
  File path
\item
  Working directory
\end{itemize}

\hypertarget{contents-1}{%
\subsection{Contents}\label{contents-1}}

*Data frames + Understanding a data frame + Count rows and columns + Add
rows and columns + Subset a data frame + list of variables + name of the
columns + Vector functions (sort, counts of values, unique values)

\begin{itemize}
\tightlist
\item
  Lists

  \begin{itemize}
  \tightlist
  \item
    Lists, as opposed to vectors, can hold components of different types
  \item
    create a list
  \item
    list subsetting
  \item
    name lists
  \end{itemize}
\item
  Matrices

  \begin{itemize}
  \tightlist
  \item
    create matrices and to understand how you can do basic computations
    with them.
  \end{itemize}
\item
  Factors

  \begin{itemize}
  \tightlist
  \item
    create, subset and compare
  \end{itemize}
\item
  Maths functions

  \begin{itemize}
  \tightlist
  \item
    Maximun value
  \item
    Minimun value
  \item
    Mean value
  \item
    Median
  \item
    Variance
  \item
    Standart deviation
  \item
    Correlation
  \item
    Round values
  \end{itemize}
\item
  Basic visualizations of the data

  \begin{itemize}
  \tightlist
  \item
    Ploting
  \item
    histograms
  \end{itemize}
\item
  Basic statistics

  \begin{itemize}
  \tightlist
  \item
    linear model
  \item
    summary of the linear model
  \end{itemize}
\item
  Using Libraries
\item
  File path
\item
  Working directory
\end{itemize}

\hypertarget{final-words}{%
\chapter{Final Words}\label{final-words}}

We have finished a nice book.

\bibliography{book.bib,packages.bib}


\end{document}
